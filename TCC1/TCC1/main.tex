\documentclass[ruledheader,noindentfirst,anapcustomindent,abntfigtabnum,tocpage=plain]{abnt}
\usepackage{amsmath, amssymb, amsthm, verbatim, amsfonts, amstext}
\usepackage[utf8]{inputenc}
\usepackage[english,portuges,brazilian]{babel}
\usepackage[T1]{fontenc}
\usepackage{graphicx}
\graphicspath{{figures/}}
\usepackage[hang,small,bf]{caption}
%\usepackage[bookmarks=false,colorlinks,linkcolor=blue]{hyperref}
\usepackage{hyperref}
\usepackage[abnt-etal-list=0,abnt-etal-text=it,abnt-and-type=&,abnt-emphasize=bf,abnt-full-initials=yes,alf,bibjustif]{abntcite}
\usepackage{cite}
\usepackage{fancyhdr}
\usepackage{algorithm}
\usepackage{algorithmic}
\usepackage{mdwlist}
\usepackage{ltxtable}
\usepackage{subfig} %Pacote para trabalhar com subfiguras
\usepackage{caption}
\usepackage{booktabs}
\usepackage{placeins}
\usepackage{tikz} %Pacote para desenhar autômatos no latex
\usetikzlibrary{arrows,automata}
\tikzstyle{width}=[draw] 
\usepackage{amssymb}
\usepackage{textcomp}
\usepackage{multirow}
\usepackage{letltxmacro}
\usepackage{relsize}
\usepackage{longtable}
\usepackage{scrextend}
\usepackage{mathtools}
\usepackage{scalefnt}
\usepackage{stmaryrd}
\usepackage[final]{pdfpages}

\definecolor{verde}{rgb}{0, .65, 0}
\definecolor{vermelho}{rgb}{1, 0, 0}
\definecolor{amarelo}{rgb}{0.9, 0.7, 0.1}
\definecolor{destaque}{rgb}{0.203, 0.65, 0.81}


\theoremstyle{definition}
\newtheorem{defn}{Definição}[section]
\newtheorem{exmp}{Exemplo}[section]
\newtheorem{restc}{Restrição}[chapter]


%\newtheorem{restc}{Restrição}[chapter]{\[}{\begin{equation*}}
%\sloppy


%%Tradução do pacote Algorithm para portugues
\renewcommand{\algorithmicrequire}{\textbf{Entrada:}}
\renewcommand{\algorithmicensure}{\textbf{Saída:}}
\renewcommand{\algorithmicend}{\textbf{fim}}
\renewcommand{\algorithmicif}{\textbf{se}}
\renewcommand{\algorithmicthen}{\textbf{então}}
\renewcommand{\algorithmicelse}{\textbf{senão}}
\renewcommand{\algorithmicelsif}{\algorithmicelse \, \algorithmicif}
\renewcommand{\algorithmicendif}{\algorithmicend \, \algorithmicif}
\renewcommand{\algorithmicfor}{\textbf{para}}
\renewcommand{\algorithmicforall}{\textbf{para todo}}
\renewcommand{\algorithmicdo}{\textbf{fazer}}
\renewcommand{\algorithmicendfor}{\algorithmicend \, \algorithmicfor}
\renewcommand{\algorithmicwhile}{\textbf{enquanto}}
\renewcommand{\algorithmicendwhile}{\algorithmicend \, \algorithmicwhile}
\renewcommand{\algorithmicloop}{\textbf{laço}}
\renewcommand{\algorithmicendloop}{\algorithmicend \, \algorithmicloop}
\renewcommand{\algorithmicrepeat}{\textbf{repetir}}
\renewcommand{\algorithmicuntil}{\textbf{até}}
\renewcommand{\algorithmiccomment}[1]{\{#1\}}
\renewcommand{\listalgorithmname}{Lista de Algoritmos}
\renewcommand{\algorithmicreturn}{\textbf{retornar}}
\floatname{algorithm}{Algoritmo}
%%%%%%%%%%%%%%%%%%%%%%%%%%%%%%%%%%%%%%%%%%%%%%%%%%%%%%%%%%%%%%%%%%%%%%%%%%%%%%%%%%%

\makeindex

%%%% O arquivo modelosCAP.tex possui as definições para ciação do estilo de capítulo (fonte de título, barras horizontais, etc.)
% ele não gera texto de saída, é um arquivo de configuração somente
%
\input{modelosCAP}
%%%%%%%%%%%%%%%%%%%%%%%%%%%%%%%%%%%%%%%%%%%%%%%FIM DO PREAMBULO%%%%%%%%%%%%%%%%%%%%%%%%%%%%%%%%%%%%%%%%%%%%%%%%%%%%%%%%%%%%%%%%%%


\begin{document}

%%%%% IMPORTANTE: ALTERA O TEXTO ENTRE ARIAL E TIMES NEW ROMAN (ALTERNAR OS COMENTÁRIOS)
%
%%%%%%%%%%%%%%%%%%%%%PARA UTILIZAR ARIAL%%%%%%%%%%%%%%%%%%%%%%%
%
\fontfamily{phv}                    %fonte Arial
\renewcommand{\rmdefault}{phv}      %
%
%%%%%%%%%%%%%%%%%%%%%PARA UTILIZAR TIMES%%%%%%%%%%%%%%%%%%%%%%%
%
%\fontfamily{ptm}               %fonte Times
%\renewcommand{\rmdefault}{ptm} %
%
%%%%%%%%%%%%%%%%%%%%%%%%%%%%%%%%%%%%%%%%%%%%%%%%%%%%%%%%%%%%%%%

%%%%%%%%%%%%%Arquivos .tex com os elementos pré-textuais
%
\thispagestyle{empty}

\vfill
 \begin{center}
    \begin{figure}[t]
     \centering
            \includegraphics[width=5cm]{./figures/IF_logo.png}\\[-0.1in]
     \end{figure}

    {\large\bfseries INSTITUTO FEDERAL DE EDUCAÇÃO, CIÊNCIA E TECNOLOGIA DO CEARÁ} \\
    {\large\bfseries PRÓ-REITORIA DE ENSINO} \\
    {\large\bfseries DEPARTAMENTO DE COMPUTAÇÃO DO CAMPUS MARACANAÚ}  \\ 
    {\large\bfseries BACHARELADO EM CIÊNCIA DA COMPUTAÇÃO}  \\ 

    \vspace*{1in}
    \begin{large} \bfseries AUREA HELENA DE SOUZA GABRIEL PEIXOTO\end{large}\\[0.4in]

    \vspace*{4cm}
    \noindent \\
    \large\bfseries{PROCESSOS DE TESTES NAS METODOLOGIAS AGEIS - UMA REVISAO BIBLIOGRAFICA } \\
    \vfill
    \large\bfseries{ MARACANAÚ \\ 2020}
\end{center}

\normalsize

\begin{titlepage}
\vfill
\begin{center}

    {\large ÁUREA HELENA DE SOUZA GABRIEL PEIXOTO\\}
    \vspace{2cm}
    {\Large \textsc{PROCESSOS DE TESTES NAS METODOLOGIAS ÁGEIS}\\}
    \vspace{1cm}
    \hspace{.45\linewidth}
    \begin{minipage}{.50\linewidth}

            Monografia submetida à Coordenadoria do Curso de Bacharelado em Ciência da Computação do Instituto Federal do Ceará - Campus Maracanaú, como requisito 
            parcial para obtenção do grau de Bacharel em Ciência da Computação.

            \vspace{0.5 cm}

            Área de pesquisa: Engenharia de Software

            \vspace{0.5 cm}

            Orientador: Prof. Me. Fabiana Marinho
    
    \end{minipage}

    \vspace{2cm}
    \vfill
    {\large Maracanaú\\ 2020}
\end{center}

\end{titlepage}

%\includepdf[pages=-, offset=0 0]{fichacatalografica.pdf}
\include{dedicatoria}
\chapter*{Agradecimentos}

Graças a Deus.


%%\thispagestyle{empty}


\begin{flushright}
\begin{minipage}[r]{10cm}
\vspace{15cm}
Dê a todas pessoas seus ouvidos, mas a poucas a sua voz.
\begin{flushright}
	William Shakespeare
\end{flushright}
\end{minipage}
\end{flushright}



%\pagestyle{plain}%%%%% Utilizar ESTILO PLAIN AQUI%%%%%%%
\chapter*{Resumo}

O texto aqui apresentado descreve o resultado inicial de uma revisão bibliográfica realizada sobre o uso das técnicas de teste nos métodos ágeis. De acordo com as leituras realizadas, foi possível perceber que a presença de testes ágeis durante o processo de desenvolvimento de software contribui para o aumento da qualidade do produto final. Na Seção 1, é descrito um contexto geral do trabalho. Na Seção 2, são discutidas as características das metodologias de testes usadas nos métodos ágeis e na Seção 3, as considerações finais e conclusões do estudo realizado são apresentadas.  

\chapter*{Abstract}

O presente texto representa o resultado de leituras realizadas a respeito da utilização das técnicas de teste utilizadas nos métodos ágeis, onde foi possível perceber que a presença de testes ágeis durante o processo de desenvolvimento de software contribui para o aumento da qualidade do produto final. 


%%%Comandos para criação automática das listas
%
\tableofcontents
%\listoffigures
%\listoftables

%%%Comandos para criar outras listas não suportadas pelo pacote ABNTex%%%
%
%\pretextualchapter{Lista de Símbolos}
%\input{nomenclatura}
%\newpage

%\pretextualchapter{Lista de Abreviações}
%
\begin{basedescript}{\desclabelstyle{\pushlabel}\desclabelwidth{6em}}
\item[{TDD}] Test-Driven Development%
\item[{BDD}] Behaviour-Driven Development%
\item[{ATDD}] Acceptance Test-Driven Development%
\end{basedescript}


%\newpage
%%%%%%%%%%%%%%%%%%%%%%%%%%%%%%%%%%%%%%%%%%%%%%%%%%%%%%%%%%%%%%%%%%%%

%Capítulos passam a ter páginas numeradas
%
\pagestyle{fancy}

%resseta os contadores de capítulo e seção
\renewcommand{\chaptermark}[1]{\markboth{#1}{}}
\renewcommand{\sectionmark}[1]{\markright{\thesection\ #1}}

%%%%%%%%%%%%%%NÃO LEMBRO O QUE FAZ, APARENTEMENTE NADA, TESTAR DEPOIS
%\fancyhf{}%
%\fancyhead[RO,LE]{\large\slshape\thepage}%
%\fancyhead[CE]{\large\slshape\leftmark}%
%\fancyhead[CO]{\large\slshape\rightmark}%


%%% Outros arquivos .tex. É acoselhável utilizar vários arquivos, pelo menos um por capítulo
%\thispagestyle{empty}


\begin{flushright}
\begin{minipage}[r]{10cm}
\vspace{15cm}
Dê a todas pessoas seus ouvidos, mas a poucas a sua voz.
\begin{flushright}
	William Shakespeare
\end{flushright}
\end{minipage}
\end{flushright}



\chapter{Introdução}\label{CAP:introducao}
\section{Contextualização}
O processo de teste é uma fase de grande importância no ciclo de vida do software. O teste permite fazer uma análise do desenvolvimento do produto e verificar se o sistema está correto com relação à sua especificação e atende às expectativas do cliente. 

A forma de execução dos testes passou por mudanças na medida que novos modelos de processo e metodologias de desenvolvimento de software foram propostos. Nas metodologias tradicionais, o teste é a última fase do processo de desenvolvimento o que pode ser um problema, pois se as especificações do produto não forem identificadas e compreendidas corretamente desde o princípio, pode ser necessário revisar todas as fases, desde a elicitação dos requisitos até o código. Como consequência, gerar atraso da entrega do software, elevando o seu custo e aumentando a insatisfação do cliente. 

Em metodologias ágeis, a maior prioridade é desenvolver um produto final que satisfaça às necessidades dos clientes. Nesse sentido, é incentivado o uso de técnicas de teste no início do processo de desenvolvimento, minimizando a possibilidade de retrabalho sem diminuir a qualidade. 
No método eXtreme Programming (XP), por exemplo, foi introduzida a metodologia Test Driven Development (TDD). Ao adotar a TDD, um teste automatizado deve ser escrito antes da implementação do programa, também denominado Test-First. Inicialmente, a TDD pode parecer complexa, já que ainda não existe código funcional escrito. No entanto, o uso da TDD exige uma maior concentração e melhor entendimento dos requisitos do produto. 

As metodologias de testes Behavior Driven Development (BDD) e Acceptance Test-Driven Development (ATDD), por sua vez, foram desenvolvidas a partir dos princípios da TDD. Sendo criados para contornar algumas falhas de comunicação identificadas com a prática da TDD.

Este trabalho busca responder a seguinte questão de pesquisa : As técnicas de testes como TDD, BDD e ATDD podem ser aplicadas a qualquer contexto de projeto que utilize alguma metodologia ágil de software.


\section{Objetivo Geral}

O presente texto possui como objetivo geral revisar, por meio de uma pesquisa bibliográfica, as técnicas de testes propostas nas metodologias ágeis.

\section{Objetivo Específicos}

Essa pesquisa procura descrever e analisar as características e diferenças entre as diversas técnicas de testes em metodologias ágeis. Com base neste objetivo será identificada a aplicabilidade dessas técnicas em diferentes tipos de projetos de software. Além disso, será verificada como a escolha da técnica influencia na qualidade do teste e projeto. Outro ponto a ser analisado consiste em investigar com que frequencia cada técnica de teste é utilizada. E por fim, pesquisar como as técnicas podem ser combinadas, a fim de otimizar a qualidade do teste.

\section{Hipótese}
Decorrente do problema de pesquisa, algumas hipóteses nortearão o desenvolvimento do estudo proposto :

\begin{itemize}
    \item Equipes de desenvolvimento com pouca familiaridade com o assunto terão mais dificuldade de aplicar as técnicas de teste nas metodologias ágeis ao projeto de software.
    \item O teste só será bem-sucedido em projetos nos quais a equipe tenha experiência suficiente e conhecimento fundamentado sobre o conceito das técnicas utilizadas nas metodologias ágeis.
\end{itemize}



\section{Justificativa}

Justifica-se esta pesquisa pois pode-se observar que com o passar dos anos os produtos de software ficam cada vez mais complexos e exigem que a qualidade do teste seja cada vez melhor. Portanto, a escolha indevida da técnica ou da falta de entendimento de sua aplicação pode afetar de forma significativa a qualidade do teste bem como no produto final.

\section{Metodologia}

Neste trabalho será adotado como metodologia a revisão bibliográfica. Segundo \cite{gil2002}, existem trabalhos construídos somente de fontes bibliográfica. Estas podem ser classificadas em livros de leitura corrente ou de referência, publicações periódicas como jornais e revistas, e impressos diversos.

Pretende-se utilizar a revisão narrativa que é um dos tipos de revisão de literatura. Desta forma, todo o conhecimento necessário para sua conclusão será adquirido mediante estudo e analise de uma vasta literatura. Com intuito de investigar como o tema proposto no trabalho se encontra documentado nos últimos anos bem como para concluir os objetivos específicos desta pesquisa. Para o levantamento dos dados será adotado como critério a inclusão de artigos científicos que tenham sido publicados entre os anos de 2005 a 2020.

As fontes de pesquisa utilizadas baseiam-se nas seguintes base de dados : Google Scholar, ResearchGate, IEEExplore e Springer. A revisão por meio de livros também será outra estratégia de pesquisa que contribuirá no desenvolvimento deste estudo.


\chapter{Referencial Teórico}\label{CAP:2}


Para \cite{palod2016agile} as metodologias ágeis buscam corrigir o principal problema existente nas técnicas tradicionais de teste que envolve a dificuldade de manutenção do código. Segundo os autores, isso acontece devido ao fato de que neste tipo de abordagem os testes são executados somente no final do processo de desenvolvimento, após a escrita do código de produção. 

A TDD é uma metodologia de teste proposta nos métodos ágeis. De acordo com a TDD, os testes devem ser implementados antes do código funcional do sistema, ou seja, ela inclui os testes desde a fase inicial do projeto. A princípio, os conceitos dessa metodologia podem ser complicados, no entanto, ajuda os desenvolvedores a se concentrarem e compreenderem melhor os requisitos \cite{moe2019comparative}. A implementação da TDD obedece um ciclo chamado de Vermelho-Verde-Refatora \cite{bulgareli2015requisitos}. O Vermelho indica que o teste deve ser implementado e executado. Como não há nenhuma funcionalidade implementada ainda, o teste resulta em falha. Na etapa representada pelo Verde, deve ser escrito um código que passe no teste criado. Em seguida, a etapa Refatora tem como objetivo fazer uma análise da implementação e mudá-la caso necessário, objetivando sempre melhorar a qualidade do código. 

A BDD surgiu da necessidade de explicar melhor o funcionamento do teste de software. Explicar a metodologia TDD para programadores que não conhecem o conceito de teste, como testar, quando testar e o que uma falha de teste significa, pode ser uma tarefa difícil \cite{bulgareli2015requisitos}. Pensando nisso, a BDD foi criada com intuito de fazer com que todas as partes interessadas (não apenas os programadores) pensassem no comportamento do sistema ao invés da sua implementação. Dessa forma, a equipe de desenvolvimento deve criar testes baseados em cenários descritos por meio da análise da especificação antes de criar o código em si. Essa característica faz que todos os envolvidos consigam ter uma melhor compreensão dos requisitos. 

A BDD permite que pessoas da equipe que não tenham nenhum conhecimento técnico entendam melhor o processo de teste \cite{barus2019implementaion}. Devido os testes serem escritos em uma linguagem simples, a curva de aprendizagem é mais curta \cite{moe2019comparative}. A BDD segue o mesmo ciclo de funcionamento da TDD, mas utiliza uma linguagem mais simples para criar as funções de testes. 

A metodologia ATDD é similar à TDD e à BDD, pois não é voltada apenas para os desenvolvedores, mas toda equipe deve colaborar e definir os chamados critérios de aceitação antes do desenvolvimento do código. Tanto a BDD quanto a ATDD são abordagens que visam envolver o cliente em toda fase do ciclo de desenvolvimento de software \cite{barus2019implementaion}. 

\cite{moe2019comparative}  faz um estudo comparativo entre a TDD, a BDD e a ATDD. O autor descreve cada uma das metodologias de forma simples e resumida, lista as vantagens e desvantagens de se utilizar cada uma, além de pontuar as diferenças entre elas. 
Segundo Moe , uma caraterística comum nas três metodologias é que os testes são executados e o código é refatorado de forma contínua. Além disso, Moe enfatiza que cada técnica é focada na perspectiva de um ou mais integrantes da equipe de desenvolvimento, sejam eles desenvolvedores, testadores ou clientes. E que elas foram criadas para suprir o déficit de entendimento dos requisitos que pode afetar a qualidade do produto final. Ainda de acordo com Moe, a TDD contribui para que os desenvolvedores foquem nos requisitos do produto, enquanto a BDD e ATDD, devido à sua linguagem não técnica permite uma melhor comunicação, colaboração e compreensão dos requisitos entre vários membros da equipe.

\cite{manuaba2019combination}, aborda a implementação da T-BDD, uma  combinação da TDD com a BDD. Além disso, o autor descreve uma análise comparativa da T-BDD com a TDD. O sistema Vixio, uma plataforma web na qual as pessoas podem escrever e compartilhar histórias de ficção interativas, foi utilizada para que a realização da análise comparativa entre A T-BDD e aTDD. O estudo conclui que, ao combinar ambas as técnicas, obtêm-se uma melhoria nos testes. O uso das metodologias em conjunto obteve um ótimo desempenho atingindo uma alta cobertura de testes. Outro resultado obtido com esse estudo é que foram observados problemas ao utilizar apenas a metodologia TDD no teste da aplicação Vixio. A TDD não funcionou bem quando houve a necessidade de mudar os parâmetros de recursos do sistema, como por exemplo, modificar os dados do banco de dados. Enquanto o teste TDD apresentou falhas após a mudança no banco de dados, o T-BDD não retornou nenhum erro.  

\cite{barus2019implementaion} no seu artigo discute a aplicação dos métodos ATDD e BDD para testar dois aplicativos web desenvolvidos por alunos do Instituto de Tecnologia Del, na Indonésia. Para aplicar os conceitos da BDD e da ATDD no projeto duas ferramentas de automação de teste foram utilizadas: Robot e Cucumber. O intuito desse estudo está em verificar se as metodologias de teste BDD e ATDD são eficazes e se ajudam a melhorar o desenvolvimento de software. 

Os resultados obtidos foram positivos, pois mostraram que as metodologias são eficazes e auxiliam na eliminação de erros o mais rápido possível. A fase de experimentos consistiu na coleta de requisitos, na escrita do teste de aceitação, na escrita do teste unitário com falha, na execução do teste e na refatoração do código. Durante as fases de testes foram encontrados alguns bugs em ambas as aplicações. Por meio do estudo, o autor também concluiu que a ATDD e a BDD envolvem o cliente em toda a fase do ciclo de vida de software e que os frameworks Cucumber e Robot funcionam bem em conjunto com as metodologias de testes que são foco do estudo.


%\chapter{Considerações finais}\label{CAP:3}

Conclui-se que os testes não servem somente para detectar e corrigir erros  de  implementação do código funcional. É uma fase também para saber se o sistema está funcionando de acordo com que foi levantado nos requisitos. E para que isso ocorra é preciso da participação de toda equipe, muito além de somente pessoas com conhecimentos técnicos. É preciso incluir pessoas de negócios nesta importante etapa. Os testes nas metodologias ágeis são usados com o intuito de manter toda a equipe de desenvolvimento do projeto focada e envolvida com os requisitos, através de práticas que possibilite uma maior compreensão das especificações do produto e aumente a comunicação entre todas as partes interessadas no projeto para que o produto final esteja de acordo com as necessidades do cliente, que é um dos princípios fundamentais da metodologia ágil \apud{nguyen2020principles}{beedle2001}.


%%%% Estilo de citação ABNT e arquivo de bibitens (mybibliography.bib)
\bibliographystyle{abnt-alf}
\bibliography{mybibliography}

%\apendice
%\include{appendices}


\end{document}

