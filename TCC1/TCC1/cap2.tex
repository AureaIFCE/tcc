\chapter{Referencial Teórico}\label{CAP:2}


Para \cite{palod2016agile} as metodologias ágeis buscam corrigir o principal problema existente nas técnicas tradicionais de teste que envolve a dificuldade de manutenção do código. Segundo os autores, isso acontece devido ao fato de que neste tipo de abordagem os testes são executados somente no final do processo de desenvolvimento, após a escrita do código de produção. 

A TDD é uma metodologia de teste proposta nos métodos ágeis. De acordo com a TDD, os testes devem ser implementados antes do código funcional do sistema, ou seja, ela inclui os testes desde a fase inicial do projeto. A princípio, os conceitos dessa metodologia podem ser complicados, no entanto, ajuda os desenvolvedores a se concentrarem e compreenderem melhor os requisitos \cite{moe2019comparative}. A implementação da TDD obedece um ciclo chamado de Vermelho-Verde-Refatora \cite{bulgareli2015requisitos}. O Vermelho indica que o teste deve ser implementado e executado. Como não há nenhuma funcionalidade implementada ainda, o teste resulta em falha. Na etapa representada pelo Verde, deve ser escrito um código que passe no teste criado. Em seguida, a etapa Refatora tem como objetivo fazer uma análise da implementação e mudá-la caso necessário, objetivando sempre melhorar a qualidade do código. 

A BDD surgiu da necessidade de explicar melhor o funcionamento do teste de software. Explicar a metodologia TDD para programadores que não conhecem o conceito de teste, como testar, quando testar e o que uma falha de teste significa, pode ser uma tarefa difícil \cite{bulgareli2015requisitos}. Pensando nisso, a BDD foi criada com intuito de fazer com que todas as partes interessadas (não apenas os programadores) pensassem no comportamento do sistema ao invés da sua implementação. Dessa forma, a equipe de desenvolvimento deve criar testes baseados em cenários descritos por meio da análise da especificação antes de criar o código em si. Essa característica faz que todos os envolvidos consigam ter uma melhor compreensão dos requisitos. 

A BDD permite que pessoas da equipe que não tenham nenhum conhecimento técnico entendam melhor o processo de teste \cite{barus2019implementaion}. Devido os testes serem escritos em uma linguagem simples, a curva de aprendizagem é mais curta \cite{moe2019comparative}. A BDD segue o mesmo ciclo de funcionamento da TDD, mas utiliza uma linguagem mais simples para criar as funções de testes. 

A metodologia ATDD é similar à TDD e à BDD, pois não é voltada apenas para os desenvolvedores, mas toda equipe deve colaborar e definir os chamados critérios de aceitação antes do desenvolvimento do código. Tanto a BDD quanto a ATDD são abordagens que visam envolver o cliente em toda fase do ciclo de desenvolvimento de software \cite{barus2019implementaion}. 

\cite{moe2019comparative}  faz um estudo comparativo entre a TDD, a BDD e a ATDD. O autor descreve cada uma das metodologias de forma simples e resumida, lista as vantagens e desvantagens de se utilizar cada uma, além de pontuar as diferenças entre elas. 
Segundo Moe , uma caraterística comum nas três metodologias é que os testes são executados e o código é refatorado de forma contínua. Além disso, Moe enfatiza que cada técnica é focada na perspectiva de um ou mais integrantes da equipe de desenvolvimento, sejam eles desenvolvedores, testadores ou clientes. E que elas foram criadas para suprir o déficit de entendimento dos requisitos que pode afetar a qualidade do produto final. Ainda de acordo com Moe, a TDD contribui para que os desenvolvedores foquem nos requisitos do produto, enquanto a BDD e ATDD, devido à sua linguagem não técnica permite uma melhor comunicação, colaboração e compreensão dos requisitos entre vários membros da equipe.

\cite{manuaba2019combination}, aborda a implementação da T-BDD, uma  combinação da TDD com a BDD. Além disso, o autor descreve uma análise comparativa da T-BDD com a TDD. O sistema Vixio, uma plataforma web na qual as pessoas podem escrever e compartilhar histórias de ficção interativas, foi utilizada para que a realização da análise comparativa entre A T-BDD e aTDD. O estudo conclui que, ao combinar ambas as técnicas, obtêm-se uma melhoria nos testes. O uso das metodologias em conjunto obteve um ótimo desempenho atingindo uma alta cobertura de testes. Outro resultado obtido com esse estudo é que foram observados problemas ao utilizar apenas a metodologia TDD no teste da aplicação Vixio. A TDD não funcionou bem quando houve a necessidade de mudar os parâmetros de recursos do sistema, como por exemplo, modificar os dados do banco de dados. Enquanto o teste TDD apresentou falhas após a mudança no banco de dados, o T-BDD não retornou nenhum erro.  

\cite{barus2019implementaion} no seu artigo discute a aplicação dos métodos ATDD e BDD para testar dois aplicativos web desenvolvidos por alunos do Instituto de Tecnologia Del, na Indonésia. Para aplicar os conceitos da BDD e da ATDD no projeto duas ferramentas de automação de teste foram utilizadas: Robot e Cucumber. O intuito desse estudo está em verificar se as metodologias de teste BDD e ATDD são eficazes e se ajudam a melhorar o desenvolvimento de software. 

Os resultados obtidos foram positivos, pois mostraram que as metodologias são eficazes e auxiliam na eliminação de erros o mais rápido possível. A fase de experimentos consistiu na coleta de requisitos, na escrita do teste de aceitação, na escrita do teste unitário com falha, na execução do teste e na refatoração do código. Durante as fases de testes foram encontrados alguns bugs em ambas as aplicações. Por meio do estudo, o autor também concluiu que a ATDD e a BDD envolvem o cliente em toda a fase do ciclo de vida de software e que os frameworks Cucumber e Robot funcionam bem em conjunto com as metodologias de testes que são foco do estudo.

