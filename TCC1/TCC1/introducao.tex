\chapter{Introdução}\label{CAP:introducao}
\section{Contextualização}
O processo de teste é uma fase de grande importância no ciclo de vida do software. O teste permite fazer uma análise do desenvolvimento do produto e verificar se o sistema está correto com relação à sua especificação e atende às expectativas do cliente. 

A forma de execução dos testes passou por mudanças na medida que novos modelos de processo e metodologias de desenvolvimento de software foram propostos. Nas metodologias tradicionais, o teste é a última fase do processo de desenvolvimento o que pode ser um problema, pois se as especificações do produto não forem identificadas e compreendidas corretamente desde o princípio, pode ser necessário revisar todas as fases, desde a elicitação dos requisitos até o código. Como consequência, gerar atraso da entrega do software, elevando o seu custo e aumentando a insatisfação do cliente. 

Em metodologias ágeis, a maior prioridade é desenvolver um produto final que satisfaça às necessidades dos clientes. Nesse sentido, é incentivado o uso de técnicas de teste no início do processo de desenvolvimento, minimizando a possibilidade de retrabalho sem diminuir a qualidade. 
No método eXtreme Programming (XP), por exemplo, foi introduzida a metodologia Test Driven Development (TDD). Ao adotar a TDD, um teste automatizado deve ser escrito antes da implementação do programa, também denominado Test-First. Inicialmente, a TDD pode parecer complexa, já que ainda não existe código funcional escrito. No entanto, o uso da TDD exige uma maior concentração e melhor entendimento dos requisitos do produto. 

As metodologias de testes Behavior Driven Development (BDD) e Acceptance Test-Driven Development (ATDD), por sua vez, foram desenvolvidas a partir dos princípios da TDD. Sendo criados para contornar algumas falhas de comunicação identificadas com a prática da TDD.

Este trabalho busca responder a seguinte questão de pesquisa : As técnicas de testes como TDD, BDD e ATDD podem ser aplicadas a qualquer contexto de projeto que utilize alguma metodologia ágil de software.


\section{Objetivo Geral}

O presente texto possui como objetivo geral revisar, por meio de uma pesquisa bibliográfica, as técnicas de testes propostas nas metodologias ágeis.

\section{Objetivo Específicos}

Essa pesquisa procura descrever e analisar as características e diferenças entre as diversas técnicas de testes em metodologias ágeis. Com base neste objetivo será identificada a aplicabilidade dessas técnicas em diferentes tipos de projetos de software. Além disso, será verificada como a escolha da técnica influencia na qualidade do teste e projeto. Outro ponto a ser analisado consiste em investigar com que frequencia cada técnica de teste é utilizada. E por fim, pesquisar como as técnicas podem ser combinadas, a fim de otimizar a qualidade do teste.

\section{Hipótese}
Decorrente do problema de pesquisa, algumas hipóteses nortearão o desenvolvimento do estudo proposto :

\begin{itemize}
    \item Equipes de desenvolvimento com pouca familiaridade com o assunto terão mais dificuldade de aplicar as técnicas de teste nas metodologias ágeis ao projeto de software.
    \item O teste só será bem-sucedido em projetos nos quais a equipe tenha experiência suficiente e conhecimento fundamentado sobre o conceito das técnicas utilizadas nas metodologias ágeis.
\end{itemize}



\section{Justificativa}

Justifica-se esta pesquisa pois pode-se observar que com o passar dos anos os produtos de software ficam cada vez mais complexos e exigem que a qualidade do teste seja cada vez melhor. Portanto, a escolha indevida da técnica ou da falta de entendimento de sua aplicação pode afetar de forma significativa a qualidade do teste bem como no produto final.

\section{Metodologia}

Neste trabalho será adotado como metodologia a revisão bibliográfica. Segundo \cite{gil2002}, existem trabalhos construídos somente de fontes bibliográfica. Estas podem ser classificadas em livros de leitura corrente ou de referência, publicações periódicas como jornais e revistas, e impressos diversos.

Pretende-se utilizar a revisão narrativa que é um dos tipos de revisão de literatura. Desta forma, todo o conhecimento necessário para sua conclusão será adquirido mediante estudo e analise de uma vasta literatura. Com intuito de investigar como o tema proposto no trabalho se encontra documentado nos últimos anos bem como para concluir os objetivos específicos desta pesquisa. Para o levantamento dos dados será adotado como critério a inclusão de artigos científicos que tenham sido publicados entre os anos de 2005 a 2020.

As fontes de pesquisa utilizadas baseiam-se nas seguintes base de dados : Google Scholar, ResearchGate, IEEExplore e Springer. A revisão por meio de livros também será outra estratégia de pesquisa que contribuirá no desenvolvimento deste estudo.

