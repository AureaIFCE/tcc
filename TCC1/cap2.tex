\chapter{Uma pequena revisão bibliográfica}\label{CAP:2}


Para \cite{palod2016agile} as metodologias ágeis buscam corrigir alguns problemas das ferramentas tradicionais de teste tais como dificuldade de manutenção do código. Isso acontece devido ao fato de que neste tipo de abordagem os testes são executados somente próximo ao final do projeto,  após a escrita do código de produção.

O desenvolvimento guiado por testes (TDD) é uma técnica utilizada durante a fase de desenvolvimento de software onde os testes devem ser implementados antes do código funcional do sistema, ou seja, inclui os testes desde a fase inicial do projeto. A princípio pode parecer um pouco complicado mas ajuda os desenvolvedores a se concentrarem nos requisitos \cite{moe2019comparative}. A implementação do TDD obedece um ciclo chamado de vermelho-verde-refatora \cite{bulgareli2015requisitos}. Vermelho indica que o teste deve ser implementado e executado. Como não há nenhuma funcionalidade implementada ainda, o teste resulta em falha. Na etapa representada pelo verde deve ser escrito um código que passe no teste criado. Logo após, a refatoração serve para fazer uma análise da implementação e mudá-la caso necessário, objetivando sempre melhorar o código atual.

O desenvolvimento guiado por comportamento (BDD) surgiu da necessidade de explicar o funcionamento do teste de software em si. Para quem não sabe o que é um teste, como testar, quando testar e o que uma falha de teste significa, tentar explicar o TDD pode ser uma tarefa difícil \cite{bulgareli2015requisitos}. Pensando nisso, o BDD foi criado com intuito de fazer com que  todas as partes interessadas no projeto pensassem no comportamento do sistema ao invés da sua implementação. Logo, a equipe de desenvolvimento deve criar testes baseados em cenários descritos por meio da análise da especificação antes de criar o código em si. Essa característica faz que todos os envolvidos consigam ter uma melhor compreensão dos requisitos. O BDD permite que pessoas da equipe que não tenham nenhum conhecimento técnico entenda melhor o processo de teste \cite{barus2019implementaion}. Devido os testes serem escritos em uma linguagem simples, a curva de aprendizagem é mais curta (Moe, 2019). O BDD segue o mesmo ciclo de funcionamento do TDD mas utiliza uma linguagem mais simples para criar as funções de testes.

O ATDD é muito similiar ao TDD e ao BDD, pois é voltado para o lado do desenvolvedor mas toda equipe deve colaborar a escolher juntos os chamados critérios de aceitação antes do desenvolvimento do código. Tanto o BDD quanto o ATDD são abordagens que visam envolver o cliente em toda fase do ciclo de desenvolvimento de software \cite{barus2019implementaion}.

\cite{moe2019comparative} faz um estudo comparativo entre o TDD, BDD e ATDD. O autor descreve cada uma das metodologias de forma simples e resumida, lista as vantagens e desvantagens de se utilizar cada uma, além de pontuar as diferenças entre elas. Uma das caraterísticas em comum nos três métodos é que há uma contínua execução de testes e refatoração de códigos. Por meio do artigo é possível também compreender que cada técnica é focada na perspectiva de um ou mais integrantes da equipe de desenvolvimento, sejam eles desenvolvedores, testadores ou clientes. E que elas foram criadas para suprir algum deficit de entendimento dos requisitos do sistema que possam afetar na qualidade do mesmo. Segundo Moe o TDD contribui para que os desenvolvedores foquem nos requisitos do produto enquanto o BDD e ATDD devido a sua linguagem não técnica permite uma melhor comunicação, colaboração e compreensão dos requisitos entre vários membros da equipe. 

\cite{manuaba2019combination} aborda a implementação do T-BDD, combinação do TDD com o BDD.  Além disso o autor faz uma análise comparativa com o TDD. O sistema Vixio, uma plataforma web onde as pessoas podem escrever e compartilhar histórias de ficção interativas, foi utilizada com o intuito de fazer as comparações entre os dois tipos de testes citados. O estudo concluiu que ao combinar ambas as tecnicas obtêm-se uma melhoria nos testes. O uso das metodologias em conjunto obteve um ótimo desempenho atingindo uma alta cobertura de testes. Pode-se observar que houve problemas ao utilizar apenas a metodologia TDD a aplicação Vixio. Este método de teste não funcionou bem quando houve a necessidade de mudar os parâmetros de recursos do sistema, como por exemplo, modificar os dados do banco de dados. Enquanto o teste TDD apresentou falhas após a mudança no banco de dados, o T-BDD não retornou nenhum erro. 
\cite{barus2019implementaion} no seu artigo discute a aplicação dos métodos ATDD e BDD para testar dois aplicativos web desenvolvidos por alunos do Instituto de Tecnologia Del, na Indonésia. Para aplicar o conceito de BDD e ATDD no projeto duas ferramentas de automação de teste foram utilizadas: Robot e Cucumber. O intuito é descrever se os métodos de testes são eficazes e se ajudam a melhorar o desenvolvimento de software. Os resultados mostraram que os métodos são eficazes e auxiliam na eliminação de erros o mais rápido possível. A fase de experimentos consistiu na coleta de requisitos, na escrita do teste de aceitação, escrever o teste unitário com falha, executar o teste e refatorar o código. Durante as fases de testes foram encontrados alguns bugs em ambas as aplicações. Por meio do estudo o autor também pode concluir que ATDD e BDD envolvem o cliente em toda a fase do ciclo de vida de software e que os frameworks Cucumber e Robot funcionaram muito bem em conjunto com os métodos de testes que são foco do estudo.
