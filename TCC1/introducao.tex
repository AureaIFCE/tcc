\chapter{Introdução}\label{CAP:introducao}
\section{Contextualização}
O processo de teste é uma fase de grande importância no ciclo de vida do software. O teste permite fazer uma análise do desenvolvimento do produto e verificar se o sistema está correto com relação à sua especificação e atende às expectativas do cliente. 

A forma de execução dos testes passou por mudanças na medida que novos modelos de processo e metodologias de desenvolvimento de software foram propostos. Nas metodologias tradicionais, o teste é a última fase do processo de desenvolvimento o que pode ser um problema, pois se as especificações do produto não forem identificadas e compreendidas corretamente desde o princípio, pode ser necessário revisar todas as fases, desde a elicitação dos requisitos até o código. Como consequência, gerar atraso da entrega do software, elevando o seu custo e aumentando a insatisfação do cliente. 

Em metodologias ágeis, a maior prioridade é desenvolver um produto final que satisfaça às suas necessidades dos clientes. Nesse sentido, é incentivado o uso de técnicas de teste no início do processo de desenvolvimento, minimizando a possibilidade de retrabalho sem diminuir a qualidade. 
No método eXtreme Programming (XP), por exemplo, foi introduzida a metodologia Test Driven Development (TDD). Ao adotar a TDD, um teste automatizado deve ser escrito antes da implementação do programa, também denominado Test-First. Inicialmente, a TDD pode parecer complexa, já que ainda não existe código funcional escrito. No entanto, o uso da TDD exige uma maior concentração e melhor entendimento dos requisitos do produto. 

As metodologias de testes Behavior Driven Development (BDD) e Acceptance Test-Driven Development (ATDD), por sua vez, foram desenvolvidas a partir dos princípios da TDD. Sendo criados para contornar algumas falhas de comunicação identificadas com a prática da TDD.  


\section{Objetivo Geral}

O presente texto possui como objetivo geral, revisar por meio de uma pesquisa bibliográfica as técnicas de testes mais usadas nos métodos ágeis que  foram abordadas na literatura.

\section{Objetivo Específicos}

Mostrar as diferenças entre as diversas técnicas de testes ágeis.
Analisar como as técnicas de testes ágeis podem ser combinadas a fim de obter a melhoria da qualidade do teste de software.
Identificar a aplicabilidade das técnicas de teste em projetos de software.
Classificar em qual contexto de aplicação cada técnica melhor se adequa.