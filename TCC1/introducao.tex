\chapter{Introdução}\label{CAP:introducao}
\section{Contextualização}
    O processo de teste é uma fase de grande importância no ciclo de vida do software. O teste permite fazer uma análise do desenvolvimento do produto.
É o momento de saber se o sistema está funcionando corretamente e se propõe a fazer aquilo que era esperado nos requisitos. A forma como os testes vêm
sendo executados mudou com os passar dos anos. Nas metodologias tradicionais de desenvolvimento de software, ele é a última etapa do processo. Isso pode
ser um problema pois se as especificações do produto não forem bem levantadas e compreendidas desde o princípio, pode ser preciso refatorar todo o código novamente. Atrasando a entrega do software, elevando o seu custo e aumentando a insatisfação do cliente. 

    Em metodologias ágeis, onde a maior prioridade é satisfazer o cliente com um produto final que atenda a suas necessidades, é incentivado o uso de
técnicas de testes ágeis para um desenvolvimento com rapidez mas sem diminuir a qualidade. No eXtreme Programming (XP), por exemplo, foi introduzido o
TDD, no qual um teste automatizado deve ser escrito antes da implementação do programa (test-first). Isto pode parecer um pouco complicado de fazer já
que não há nenhum código funcional escrito ainda mas faz com que haja uma necessidade maior de se concentrar nos requisitos do produto. Os métodos de
testes BDD e ATDD, por sua vez, foram desenvolvidos a partir dos princípios do TDD. Sendo criados para contornar algumas falhas de comunicação deixados
por ele. 

    O presente texto busca fazer uma pequena revisão bibliográfica sobre as diferentes técnicas de testes usados nas metodologias ágeis. Aqui serão
apresentados conceitos, características e aplicações dos testes TDD, BDD e ATDD.
